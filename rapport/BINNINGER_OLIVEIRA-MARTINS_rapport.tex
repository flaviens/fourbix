\documentclass[titlepage,11pt,a4paper]{article}

\usepackage[T1]{fontenc}
\usepackage[utf8]{inputenc}
\usepackage{lmodern}
\usepackage[french]{babel}
\usepackage{url,csquotes}
\usepackage[hidelinks,hyperfootnotes=false]{hyperref}
\usepackage[titlepage,fancysections]{polytechnique}
\usepackage{amsmath}
\usepackage{amssymb}
\usepackage{mathrsfs}
\usepackage{graphicx}
\usepackage{multicol}
\usepackage{stmaryrd}
\usepackage{textcomp} %permet de faire le °
\usepackage{algorithm}
\usepackage{algorithmic}
\usepackage{listings} %permet d'inclure du java

%Ajouter ça pour le code java en couleur :

\lstset{
language=php,
basicstyle=\normalsize, % ou ça==> basicstyle=\scriptsize,
upquote=true,
aboveskip={1.5\baselineskip},
columns=fullflexible,
showstringspaces=false,
extendedchars=true,
breaklines=true,
showtabs=false,
showspaces=false,
showstringspaces=false,
identifierstyle=\ttfamily,
keywordstyle=\color[rgb]{0,0,1},
commentstyle=\color[rgb]{0.133,0.545,0.133},
stringstyle=\color[rgb]{0.627,0.126,0.941},
numbers=left,
}


\title{Fourbix}
\logo{../images/logo/accueil-logo.png}
\subtitle{INF473W - modal Web - Rapport de projet}
\author{Alexandre \bsc{Binninger} - Gabriel \bsc{Oliveira Martins}}
\date{\today}

\begin{document}

\maketitle

\clearpage

\setcounter{page}{1}


\section{Utilisation du site}

\subsection{A quoi sert ce site ?}

\emph{FourbiX} est un site de prêt de matériel par les binets. Il permet ainsi aux utilisateurs de consulter les offres des binets et de réaliser facilement leur demande en ligne. L'échange entre les binets est alors facilité et les démarches sont fluidifiées. Le nom du site est inspiré du mot d'argot militaire <<fourbi>> qui désigne l'ensemble des possessions d'un soldat.

Nous avons travaillé avec GitHub pour gérer ce projet et vous pouvez trouver le dépôt à l'adresse suivante : \url{https://github.com/gabrieloliveiragom/fourbix}. 


\subsection{Comment utiliser FourbiX ?}

Sur \emph{FourbiX}, il est possible de taper directement le nom d'un item dans la barre de recherche ou alors de rechercher un binet dans le catalogue. Les pages des binets exposent le matériel qu'ils mettent à disposition et il suffit de cliquer dessus pour réaliser une commande. Les demandes peuvent être faites à son propre nom ou au nom d'un binet auquel on appartient et on peut les consulter directement via la page dédiée "Demandes".\\

Certaines informations sur l'objet sont disponibles : évidemment son nom et la quantité restante, mais il est possible pour le membre du binet d'indiquer une courte description et de déposer une photo de l'objet qui seront affichées dans le caroussel dans le catalogue. Le binet peut fixer une caution s'il le souhaite.\\

Les Administrateurs de binet peuvent gérer les membres dudit binet. Les responsables de matériel gèrent le matériel et les demandes associées. Ceux qui sont simplement "membres" du binet peuvent faire une demande pour le nom de ce binet. Pour être administrateur du site, il faut être administrateur du binet "Administrateurs" et il est alors possible de gérer tous les binets, de rajouter des binets et des utilisateurs et d'en supprimer.\\

Un utilisateur fait une demande à un binet. Le binet peut alors accepter la demande sur sa page. Une deadline est renseignée, date à partir de laquelle le binet se donne le droit d'encaisser une caution éventuelle.

\subsection{Gestion du matériel}

La page des binets permet de gérer le matériel. Il est possible de le faire directement sur le tableau. Bien sûr, ces options de modification n'apparaissent pas pour un utilisateur normal. 


\section{Le Front-end}

\section{Le Back-end}


%TODO : on a des avertissements printés si jamais il y a echec ou réussite.
\end{document}